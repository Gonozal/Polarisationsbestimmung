% Klassifiziert den Dokumenten-Typ
% Doku: http://exp1.fkp.physik.tu-darmstadt.de/tuddesign/
% Farben: http://www.tu-darmstadt.de/media/medien_stabsstelle_km/services/medien_cd/das_bild_der_tu_darmstadt.pdf
%  bigchapter: Chapter haben doppelte Schriftgröße
%  linedtoc: Linien im Inhaltsverzeichnis wie bei Überschriften
%  colorbacktitle: Der Dokumenten-Titel wird mir der Accentfarbe hinterlegt
\documentclass[bigchapter,colorback,accentcolor=tud4b,linedtoc,11pt]{tudreport}

% Input Dokument hat das Encoding UTF-8
\usepackage[utf8]{inputenc}
% Wichtiges Paket für Links und verlinktes Inhaltsverzeichnis
\usepackage[ngerman]{hyperref}
% Paket für Fußnoten
\usepackage[stable]{footmisc}
% Paket für Bibliotheks-Verzeichnis, square: Verwende eckige statt runde klammern
% \usepackage[square]{natbib}
% Paket zum Plotten von Datensätzen
\usepackage{pgfplots}
% Verwende deutsche Bezeichner für Inhaltsverzeichnis, ... (ngerman = New German: neue Rechtschreibung)
\usepackage{ngerman}
% Deutsche Zahlen (entfernt z.B. das Leerzeichen nach einem Dezimal-Komma)
\usepackage{ziffer} 

\usepackage[verbose]{placeins}

%\usepackage{graphicx}
%\usepackage{caption}
\usepackage{subcaption} %Für subfigures

% PDF-Optionen
\hypersetup{
  pdftitle={TU Darmstadt \- Physikalisches Praktikum für Fortgeschrittene},
  pdfauthor={Esra Bauer und Sören Link},
  pdfsubject={Versuch 3.3-B},
  pdfview=FitH,
}
% Nummeriere formeln in Subsections einzeln
% Kleines makro zur assymetrischen Fehlerangabe

% Entspricht-Zeichen
\usepackage{scalerel}

\newcommand\equalhat{%
\let\savearraystretch\arraystretch
\renewcommand\arraystretch{0.3}
\begin{array}{c}
\stretchto{
    \scalerel*[\widthof{=}]{\wedge}
    {\rule{1ex}{3ex}}%
}{0.5ex}\\ 
=%
\end{array}
\let\arraystretch\savearraystretch
}
%BEGINN TITELSEITE

\title{Polarisationsanalyse von Licht mittels Stokes-Formalismus}

\subtitle{Esra Bauer  \\Sören Link}

\subsubtitle{Betreuer: David Rupp \hfill Versuchsdatum: 10. November 2014}

\author{Esra Bauer, Sören Link}

%\settitlepicture{img/title.jpg}

\institution{Physikalisches Praktikum \\für Fortgeschrittene \\ Versuch 3.3-B}

\date{\today}


%ENDE TITELSEITE

\begin{document}
%ANFANG DOKUMENT

%Titelseite einfügen
\maketitle

%Inhaltsverzeichnis einfügen
\tableofcontents

%ANFANG INHALT

\chapter{Einleitung}

In diesem Versuch wird Licht verschiedener Polarisationszustände hergestellt und untersucht mit dem Ziel, diese Zustände anschließend vollständig beschreiben zu können, das heißt sowohl für vollständig als auch für teilweise polarisiertes Licht. Wenn man von polarisiertem Licht spricht, bezieht man sich stets auf die Richtung, in der der elektrische Feldvektor oszilliert. Wir betrachten im Folgenden Licht im kartesischen Koordinatensystem, also eine transversale elektromagnetische Welle, welche sich in z-Richtung ausbreitet. Zur Einordnung der Polarisation ist folglich das Verhalten der Elektrischen Feldkomponenten $E_x$ und $E_y$ von Bedeutung.

\chapter{Grundlagen}
\section{Polarisation}


Bei unpolarisiertem Licht, beispielsweise dem Licht der Sonne oder eine Glühlampe, existiert keine feste Beziehung zwischen den Feldkomponenten $E_x$ und $E_y$ bezüglich der Phasenverschiebung $\delta$ und der Beträge $E_{i}$. Schwingt der Feldvektor in einer Ebene, das heißt gilt $\delta=0$, liegt lineare Polarisation vor. Rotiert der Feldvektor um die z-Achse, liegen eine Phasenverschiebung $\delta=\frac{\pi}{2}$ und $E_{x}=E_{y}$ zu Grunde und man spricht von zirkular polarisiertem Licht. Falls $\delta$ einen anderen, jedoch festen Wert annimmt und $E_{x}\neq E_{y}$ gilt, ist die Polarisation elliptisch. 


\section{Stokes-Formalismus}

Mittels des sogenannten Stokes-Formalismus wird es möglich, Mischformen der Polarisationszustände zu klassifizieren und auch den Grad der Polarisation zu bestimmen. Dazu werden die vier Stokes-Parameter wie folgt definiert:

\begin{center}
$S_0=P_{0} + P_{90}=E_{0x}^2 + E_{0y}^2,$\vspace{\baselineskip}

$S_1=P_{0} - P_{90}=E_{0x}^2 - E_{0y}^2,$\vspace{\baselineskip}

$S_2=P_{45} - P_{135}=2E_{0x} \dcot E_{0y} \cdot cos(\delta),$\vspace{\baselineskip}
 
$S_3=P_{rechts} - P_{links}=2E_{0x} \dcot E_{0y} \cdot sin(\delta)$\vspace{\baselineskip}
\end{center}


Das Heißt $S_0$ gibt die Gesamtintensität an, während $S_1$ den in x- oder y- Richtung linar polarisierten Anteil angibt, $S_2$ den Anteil zur x- oder y Achse um 45° Grad verdreht und $S_3$ den rechts- beziehungsweise links zirkular polarisierten Anteil. Wir schreiben die Parameter als Spaktenvektor, den sogenannten Stokes-Vektor $S$, da es sich herausgestellt hat, dass aus einzelnen Stokes-Vektoren durch vektorielle Addition sinnvolle resultierende Stokes-Vektoren ergeben, auch wenn der Stokes-Vektor im streng mathematischen Sinn kein Vektor ist.
Es geht also bei diesem Versuch im Wesentlichen darum, für verschiedene Lichtquellen unbekannter Polarisation die entsprechenden Stokes-Vektoren zu berechnen und somit die Polarisation zu bestimmen. Dazu ist es nötig, sich zunächst mit den verwendeten Lichtquellen und der Erzeugung von Polarisation zu befassen. Im Versuch werden eine weiße LED und ein oberflächenemittierender Halbleiterlaser als Lichtquelle verwendet. 


\section{Lichtemittierende Dioden und Halbleiterlaser}

Beide Lichtquellen senden Licht nach dem Prinzip von p-n-Übergängen aus, das heißt ein p-dotierter Halbleiter, dessen Ladungsträger "Löcher" sind und ein n-dotierter Halbleiter, dessen Ladungsträger Elektronen sind, stehen in Kontakt, wodurch sich eine sog. Verarmungs- bzw. Raumladungszone ausbildet, das heißt an der Kontaktstelle rekombinieren Elektronen und Löcher derart, dass sich die Fermi-Energien der beiden Materialien angleichen. Nun wird eine Spannung von außen in Durchlassrichtung angelegt, wodurch zusätzliche Elektronen in die n-dotierte Schicht eingebracht werden und zur Verarmungszone wandern, wo sie mit Löchern der p-dotierten Schicht rekombinieren und dabei Energie in Form von Licht freisetzen. Die Wellenlänge ist durch die Größe der Bandlücke bestimmt, da der Energiebetrag der Größe der Bandlücke entspricht. Das heißt derartige Lichtquellen senden im Wesentlichen monochromatisches Licht aus und um die besagte weiße LED realisieren zu können, müssen entweder mehrere Halbleiderdioden verschiedener Farben kombiniert werden oder Lumineszenzfarbstoffe auf eine blaue LED aufgebracht werden. Ein weiterer Unterschied zwischen der weißen LED und dem Laser ist der optische Resonator, den nur der Laser besitzt. Im Falle des oberflächenemittierenden Lasers ist der Resonator vertikal ausgerichtet und besteht aus zwei DBR-Spiegeln (engl.: Distributed Bragg Reflectors), zwischen denen konstruktive Interferenz stattfindet. Der Resonator ist extrem kurz (in der Größenordnung einer Wellenlänge), wodurch die Besetzungsinversion, die für das Laserprinzip erforderlich ist, bereits bei einem sehr niedrigem Schwellstrom entsteht.

%% Verhindert, dass irgend etwas, das VOR der floatbarrier steht hinter content dargestellt wird, der danach steht
%% Sehr nützlich für bilder und graphen

%% Beispiel für einen Graphen
% \begin{figure}[h]
% \centering
%                 \begin{tikzpicture}
% 								\begin{axis}[
% 								title={Stabilitätsdiagramm},
% 								xmin=-4,
% 								xmax=4,
% 								ymin=-4,
% 								ymax=4,
% 								axis y line=center,
% 								axis x line=center,
% 								%restrict y to domain=-20:20,
% 								samples=100,
% 								xlabel=$g_1$,
% 								ylabel=$g_2$,
% 								]
% 								
% 								
% 								\addplot[fill,opacity=0.3,cyan,domain=-5:5,smooth] % smooth breaks \closedcycle
%   {1/x} \closedcycle;
% 								\addplot[smooth, dashed, domain=-5:5] {x};
% 								
% 								\end{axis}
% 								\end{tikzpicture}
% 	\caption{
% 		Der blau unterlegte Bereich ist stabil. Auf der diagonalen Linie liegen alle symmetrischen Resonatoren. 
% 	}
% \end{figure}


\section{Polarisatoren und $\frac{\lambda}{4}$-Plättchen}
Beiden Lichtquellen ist gemeinsam, dass sie keine stabile Polarisation besitzen, das heißt diese muss im Versuch selbst hergestellt werden. Zu diesem Zweck bedienen wir uns des Prinzips der Doppelbrechung. Dabei wird Licht, wenn es auf ein doppelbrechendes Material trifft, in zwei Teilstrahlen aufgespalten, für die unterschiedliche Brechungsindizes gelten. Man unterscheidet den ordentlichen Strahl, dessen Polarisation linear und senkrecht zur optischen Achse des Materials (i.d.R. ein Kristall wie z.B. Calcit) ist und der dem Snellius'schen Brechungsgesetz genügt, und den außerordentlichen Strahl, der parallel zur optischen Achse polarisiert ist und dessen Brechungsindex von der Ausbreitungsrichtung relativ zur optischen Achse abhängt. Das Heißt auf diese Weise können wir linear polarisiertes Licht erzeugen. Im Speziellen arbeiten wir mit Glan-Thomson-Polarisationsprismen, die aus zwei Prismen aus doppelbrechendem Material derart zu einen Quader verbunden sind, dass die opische Achse parallel zur Eintrittsfläche liegt und der ordentliche Strahl an der schrägen Verbindungsfläche beider Prismen total reflektiert wird. Der außerordentliche Strahl tritt ohne Ablenkung wieder aus und ist nun linear polarisiert.

\vspace{\baselineskip}Um zirkular polarisiertes Licht zu erzeugen, wird linear polarisiertes Licht auf ein $\frac{\lambda}{4}$-Plättchen gestrahlt. Dieses besteht wieder aus einem doppelbrechenden Material, dessen optische Achse senkrecht zur Ausbreitungsrichtung steht. Beide Teilstrahlen, ordentlicher und außerordentlicher Strahl treten in gleicher Richtung wieder aus, jedoch mit einer Phasendifferenz von $\frac{\pi}{2}$, das heißt es entsteht zirkular polarisiertes Licht. Im Versuch werden sog. zusammengesetzte $\frac{\lambda}{4}$-Plättchen nullter Ordnung verwendet, das heißt das Licht strahlt durch zwei doppelbrechende Materialen, deren optische Achsen senkrecht zueinander ausgerichtet sind. Die Phasenverschiebung kommt nicht durch die Gesamtdicke des Plättchens zustande, sondern durch die Dickendifferenz der beiden Bestandteile. Dieser Trick ist notwendig, da sich solch geringe Dicken, wie sie für Licht notwendig wären, nicht ohne weiteres herstellen und händeln lassen.

%% Beispiel für das Einbinden von bildern
% \begin{figure}[ht!]
% \centering
% \includegraphics[width=65mm]{img/5074.png}
% \caption{Energieschema eines HeNe-Lasers}
% \label{HeNeLaser}
% \cite{HeNeNiveaus}
% \end{figure}


\section{Müller-Matrix-Formalismus}
Wie oben beschrieben, lassen sich mit Stokes Vektoren $\vec{S}= (S_0, S_1, S_2, S_3)^T$ resultierende Polarisationszustände berechnen. Das legt nahe, dass man auch Matrizen auf diese Vektoren anwenden darf. Man kann in der Tat optischen Bauteilen sog. Müller-Matrizen zuordnen, die als Transformationsmatrizen für die Stokes Vektoren dienen. Ist ein einfallender Lichtstrahl durch den Stokes-Vektor $S$ charakterisiert und ein optisches Bauteil durch die Müller-Matrix $M$, so ergibt sich der austretende Lichtstrahl zu $S' = M \cdot S$ 

\section{Drehwinkelaufnehmer}
Um den Drehwinkel des $\frac{\lambda}{4}$-Plättchens kontrollieren zu können bzw. Kennlinien über dem Drehwinkel aufnehmen zu können, steht uns ein Drehwinkelaufnehmer, in den ein $\frac{\lambda}{4}$-Plättchen eingebaut ist, zur Verfügung. Damit ist es uns möglich, den Winkel in Schritten von 5 Grad aufzunehmen, da auf der Oberfläche des Drehwinkelaufnehmers stark und schwach reflektierende Beschichtungen abwechselnd (in 5 Grad Unterteilung) aufgebracht sind, die von einer Reflexionslichtschranke als Signal verarbeitet werden. Die verwendete Software nimmt automatisch einen Leistungswert vom Detektor auf, sobald ein Signal der Lichtschranke vorliegt. 


\section{Gefahren durch Laserstrahlung und Vorsichtsmaßnahmen}
Bei dem in diesem Versuch vorliegenden VCSEL Laser mit einigen mW Leistung im roten Spektralbereich sind die potentiellen Gefahren durch Laserstrahlung überschaubar. Vor allem muss hier darauf geachtet werden, dass der Laser nicht in ein Auge gelangt. Aus diesem Grund ist bei eingeschaltetem Laser immer eine Schutzbrille zu Tragen und der Kopf ist nie auf Höhe des Lasers zu halten. Auch sollte darauf geachtet werden, dass potentielle Reflexe des Lasers wenn möglich auf die Laserapparatur selbst zurückgelengt werden und vor allem nicht auf einen Eingang zeigen, da sonst außenstehende ohne Schutzbrille gefährdet werden können.

Bei Lasern mit höherer Intensität ist zudem der Kontakt mit dem Körper zu vermeiden, da ein Laser nicht nur oberflächliche Verbrennungen, sondern im Falle eines UV-Lasers auch Hautkrebs und im Falle eines IR-Lasers schmerzlose und deswegen schwer zu erkennende Verbrennungen im Unterhautgewebe verursachen können. Zur Sichtbarmachung von Lasern sollte deswegen nie die nackte Haut sondern eine nicht reflektierende Oberfläche (beispielsweise ein Schirm, der die verwende Laserleistung aushält) verwendet werden.

Abgesehen von den genannte Personenschäden sind bei nicht sachgemäßer Handlung von Lasern mit hoher Intensität auch Schäden am Versuchsaufbau möglich. Verschmutzte Spiegel oder Fenster des Lasermediums können zu extremer Hitzeentwicklung an jeweiligen Material und letztendlich zu dessen Ermattung oder gar Zerstörung führen.
%% quellenangabe
\cite{GefahrenLaser}


\chapter{Durchführung}
\section{Aufnahme der P-I-Kennlinie der LED}
\label{sec:p-i-kennlinie}
Zuerst nehmen wir eine P-I-Kennlinie der weißen LED auf. Das heißt, wir messen die abgestrahlte Lichtleistung über dem Strom. Da die LED in einem sehr großen Winkel abstrahlt, verwenden wir hierfür zwei Linsen, um eine Kollimation zu erreichen. Unser Aufbau besteht hierbei aus der LED, den beiden Linsen und dem Detektor, die in dieser Reihenfolge auf einem Reiter montiert werden.

\begin{center}
\begin{figure}[h]
\begin{tikzpicture}
\begin{axis}[
    legend pos=south west,
    title={Leistung der weißen LED in abhängigkeit der Stromstärke},
    xlabel=Stromstärke in mA,
    ylabel=Leistung in W,
    width=0.9\textwidth,
    height= 9cm,
    xmin=0,
    xmax=55,
    grid=both,
    ymin=0,
    ymax=0.0045,
    tick align=outside,
    tickpos=left,
    minor x tick num=3,
    minor y tick num=4,
    minor grid style={dotted,thin}
]
\addplot[red, only marks, mark=x, mark size=1pt, error bars/.cd, y dir=both, y fixed relative=0.01, x dir=both, x fixed=0.05]
table[x index={0},y index={1}] {data/1.txt};
\addlegendentry{Leitung der LED auf der Photoplatte}
\end{axis}
\end{tikzpicture}
\captionof{figure}{Gemessene Leistung der LED auf der Photoplatte in abhängigkeit von der angelegten Stromstärke. Die angenommenen Fehler für die Stromstärke und die gemessene Leistung der LED betragen 0.05mA und 1 Prozent.}
\end{figure}
\end{center}

Als Fehler für die Stromstärke wurden 0.05mA angenommen, da die Stromquelle nur eine Nachkommastelle anzeigte. Dagegen wurde als Fehler für die aufgenommene Leistung 1 Prozent angenommen. Zwar ist das verwendete Messgerät deutlich genauer, allerdings wurden dennoch offensichtliche Schwankungen in der gemessenen Leistung beobachtet, welche entweder durch den im Raum verwendeten Monitor oder durch eine nicht konstante Leistung der LED verursacht wurden.

\section{Auslöschungsverhältniss der Linearpolarisatoren}
Zur Bestimmung des Auslöschungsverhältnisses der Linearpolarisatoren wurden in den Aufbau aus Abschnitt \ref{sec:p-i-kennlinie} noch 2 Linearpolarisatoren eingefügt.
Anschließend wurde einer der Polarisatoren so lange gedreht, bis die aufgenommene Leistung ein Minimum erreicht hat, die Polarisationsachsen der Polarisatoren also um 90° zueinander verdreht waren. Danach wurde der vordere Linearpolarisator um 90° gedreht und es wurde erneut die gemessene Leistung augenommen.

\begin{center}
  \begin{tabular}{|p{5cm}|p{4cm}|p{4.5cm}|}
    \hline
        Winkel zwischen den Polarisationsachsen in Grad & Leistung in W & Fehler der Leistung in W \\ \hline
        90 & $6,75 \cdot 10^-9$ & $0,05 \cdot 10^-4$ \\ \hline
        90 & $5,111 \cdot 10^-4$ & $0,05 \cdot 10^-9$ \\ \hline
	\end{tabular}
\end{center}
\section{Bestimmung der wellenlängenaufgelösten Verzögerung des achromatischen Verzögerungsplättchens}
Im Abschnitt 2.4 wurde bereits das $\frac{\lambda}{4}$-Plättchen beschrieben. Tatsächlich beträgt die Phasenverschiebung nicht immer genau $\frac{\pi}{2}$, sondern ist wellenlängenabhängig. Diese Abhängigkeit soll nun untersucht werden. Zu diesem Zweck steht uns ein Babinet-Soleil-Kompensator (im Folgen als BSK bezeichnet) zur Verfügung. Es handelt sich dabei grundsätzlich um ein verstellbares zusammengesetztes Verzögerungsplättchen nullter Ordnung, d.h. wie auch bei den verwendeten $\frac{\lambda}{4}$-Plättchen liegen zwei doppelbrechende Materialien derart aufeinander, dass die optischen Achsen senkrecht zueinander sind. Jedoch besteht beim BSK eine der Platten aus zwei Keilen, die per Stellschraube gegeneinander verschoben werden können, wodurch sich die Dickendifferenz und somit die Phasenverschiebung ändert. Da Leistungsminima am einfachsten gemessen werden können, fügen wir den BSK  zwischen beide Polarisatoren ein, die wir vorher senkrecht zueinander ausgerichtet haben (immer in Bezug auf die optischen Achsen) und richten ihn so aus, dass immernoch ein Minimum gemessen wird. Nun verdrehen wir ihn um 45 Grad und justieren die Dicke solange nach, bis drei nebeneinanderliegende Minima gemessen werden können. Diese Messungen führen wir mit verschiedenen Farbfiltern durch, jeweils bei 488nm, 543,5nm, 633nm und 694,3nm, um eine Referenz zu finden. Folgende Werte lesen wir am BSK ab, wobei wir den Fehler mit 0,06 abschätzen: 

\begin{center}
  \begin{tabular}{|c|c|c|c|}
    \hline
        Filter 1 (488nm) & Filter 2 (543,5nm) & Filter 3 (633nm) & Filter 4 (694,3nm) \\ \hline
        30,91 & 29,47 & 27,11 & 25,64 \\ \hline
        42,19 & 42,21 & 42,22 & 42,24 \\ \hline
        53,55 & 54,95 & 57,29 & 58,80 \\ \hline
	\end{tabular}
\end{center}

Anschließend wird das $\frac{\lambda}{4}$-Plättchen in den Strahlengang hinter den BSK integriert, nach dem Minimum ausgerichtet, um 45 Grad gedreht und die Messung wiederholt, was uns folgende Werte liefert:  

\begin{center}
  \begin{tabular}{|c|c|c|c|}
    \hline
        Filter 1 (488nm) & Filter 2 (543,5nm) & Filter 3 (633nm) & Filter 4 (694,3nm) \\ \hline
        27,97 & 26,18 & 23,41 & 21,60 \\ \hline
        39,33 & 38,95 & 38,48 & 38,22 \\ \hline
        50,62 & 51,67 & 53,50 & 54,81 \\ \hline
	\end{tabular}
\end{center}

Die Differenzen der am BSK eingestellten Werte mit und ohne  $\frac{\lambda}{4}$-Plättchen ergeben die Phasenverzögerung, wobei wir noch auf ganze Wellenlängen skalieren müssen, d.h. durch $2\pi$ dividieren (dies liefert $\delta$ in Anteilen der ganzen Wellenlängen) bzw. mit $\frac{360^\circ}{2\pi}}$ multiplizieren (damit berechnen wir die Phasenverschiebung im Gradmaß). Die Phasenverschiebungen in Anteilen der ganzen Wellenlängen sind also: 

\begin{center}
  \begin{tabular}{|c|c|c|c|}
    \hline
        Filter 1 (488nm) & Filter 2 (543,5nm) & Filter 3 (633nm) & Filter 4 (694,3nm) \\ \hline
        0,468 & 0,524 & 0,589 & 0,643 \\ \hline
        0,455 & 0,519 & 0,595 & 0,640 \\ \hline
        0,466 & 0,522 & 0,603 & 0,635 \\ \hline
	\end{tabular}
\end{center}

Das Messen dreier Werte ist sinnvoll, um eine größere Genauigkeit zu erreichen. Nach Mittelwertbildung und Umrechnung auf das Gradmaß ergibt sich eine Verteilung gemäß folgender Grafik:

\begin{center}
\begin{figure}[h]
\begin{tikzpicture}
\begin{axis}[
    legend pos=south west,
    title={Phasenverschiebung in Abhängigkeit der Wellenlänge},
    xlabel=Wellenlänge in nm,
    ylabel=Phasenverschiebung in Grad,
    width=0.9\textwidth,
    height= 9cm,
    xmin=450,
    xmax=730,
    grid=both,
    ymin=150,
    ymax=250,
    tick align=outside,
    tickpos=left,
    minor x tick num=3,
    minor y tick num=4,
    minor grid style={dotted,thin}
]
\addplot[red, only marks, mark=x, mark size=1pt, error bars/.cd, y dir=both, y fixed = 3,44, x dir=both, x fixed = 5]
table[x index={0},y index={1}] {data/delta.txt};
%\addlegendentry{}
\end{axis}
\end{tikzpicture}
\captionof{figure}{Wellenlängenabhängige Phasenverschiebung des Verzögerungsplättchens. Der Fehler der Wellenlänge ist 5nm (Quelle: Datenblatt), den Fehler der abgelesenen Werte am BSK haben wir mit 0,06 abgeschätzt, was einem Fehler der Phasenverschiebung von 3,44 entspricht.}
\end{figure}
\end{center}

\section{Charakterisierung des Polarisationszustandes mit Hilfe des Stokes-Formalismus}
TODO: Skizze/Aufbau, Einstellungen, aufgenommene Daten, wie wurden die verschiedenen Polarisationen hergestellt?

\section{P-I-Kennlinie der VCSEL}
TODO: Skizze/Aufbau, Einstellungen, aufgneommene Daten, anmerkung: wurde nach aufgabenteil f gemacht, Leistungsgrenze von 12mA vorgegeben

\section{Polarisationszustand des VCSLEs in Abhägigkeit des Stroms}
TODO: Skizze/Aufbau, Einstellungen, aufgneommene Daten, anmerkung: wurde vor aufgabenteil e gemacht

\chapter{Auswertung}
\section{Aufnahme der P-I-Kennlinie der LED}
TODO: Tabelle, Graph, Fehlerrechnung

\section{Auslöschungsverhältniss der Linearpolarisatoren}
TODO: Gegenüberstellung, Quotient, Fehlerrechnung

\section{Bestimmung der wellenlängenaufgelösten Verzögerung des achromatischen Verzögerungsplättchens}
TODO: Berechnung der Kalibrationsfaktoren für die einzelnen Wellenlängen, Fehlerrechnung für Kalibration; Berechnung der Verschiebung durch Verzögerungsplätchen, Umrechnung in Grad/Radian

\section{Charakterisierung des Polarisationszustandes mit Hilfe des Stokes-Formalismus}
TODO: Berechnung der Stokes-Parameter. Wichtig: Benutzung der in Aufgabenteil d bestimmten, genaueren und von 90° abweichenden, verzögerung des $\frac{\lambda}{4}$ Plättchens. Was ergibt sich für den Aufbau der 3-D Brille?

\section{P-I-Kennlinie der VCSEL}
TODO: Folerungen für den Laserschutz, Warum kein linearer Anstieg? Fehlerbalken!

\section{Polarisationszustand des VCSLEs in Abhägigkeit des Stroms}
TODO: Grad und Art der Polarisation des VCSLEs, normieren und Fehlerrechnung
\chapter{Fazit}
TODO: Kurze Zusammenfassung über Auswertung. Bewertung der gewonnenen Daten, eingehen auf eventuelle Probleme

%ENDE INHALT
\cleardoublepage{}
% Eintrag fürs Inhaltsverzeichnis
\newpage
\begin{thebibliography}{100}
  \bibitem{GefahrenLaser} \url{http://de.wikipedia.org/w/index.php?title=Laser&oldid=128632514#Gefahren}
\end{thebibliography}

\cleardoublepage{}
% Eintrag fürs Inhaltsverzeichnis
% Abbildungsverzeichnis einfügen
\end{document}
